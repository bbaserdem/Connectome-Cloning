% ****** Start of file apssamp.tex ******
%
%   This file is part of the APS files in the REVTeX 4.1 distribution.
%   Version 4.1r of REVTeX, August 2010
%
%   Copyright (c) 2009, 2010 The American Physical Society.
%
%   See the REVTeX 4 README file for restrictions and more information.
%
% TeX'ing this file requires that you have AMS-LaTeX 2.0 installed
% as well as the rest of the prerequisites for REVTeX 4.1
%
% See the REVTeX 4 README file
% It also requires running BibTeX. The commands are as follows:
%
%  1)  latex apssamp.tex
%  2)  bibtex apssamp
%  3)  latex apssamp.tex
%  4)  latex apssamp.tex
%
\documentclass[%
 reprint,
%superscriptaddress,
%groupedaddress,
%unsortedaddress,
%runinaddress,
%frontmatterverbose, 
%preprint,
%showpacs,preprintnumbers,
%nofootinbib,
%nobibnotes,
%bibnotes,
 amsmath,amssymb,
 aps,
%pra,
%prb,
%rmp,
%prstab,
%prstper,
%floatfix,
]{revtex4-1}

\usepackage{bbold}
\usepackage{graphicx}% Include figure files
\usepackage{dcolumn}% Align table columns on decimal point
\usepackage{bm}% bold math
%\usepackage{hyperref}% add hypertext capabilities
%\usepackage[mathlines]{lineno}% Enable numbering of text and display math
%\linenumbers\relax % Commence numbering lines

%\usepackage[showframe,%Uncomment any one of the following lines to test 
%%scale=0.7, marginratio={1:1, 2:3}, ignoreall,% default settings
%%text={7in,10in},centering,
%%margin=1.5in,
%%total={6.5in,8.75in}, top=1.2in, left=0.9in, includefoot,
%%height=10in,a5paper,hmargin={3cm,0.8in},
%]{geometry}

\begin{document}

%\preprint{APS/123-QED}

\title{Connectome Cloning}% Force line breaks with \\
%\thanks{}%

\author{Batuhan Başerdem}
 \altaffiliation[PhD: ]{SUNY at Stony Brook, Dept. of Physics}%Lines break automatically or can be forced with \\
 \email{bbaserde@cshl.edu}
%\author{Second Author}%
 
\affiliation{%
 Cold Spring Harbor Laboratories \textbackslash\textbackslash
}%

%\collaboration{MUSO Collaboration}%\noaffiliation


\collaboration{CLEO Collaboration}%\noaffiliation

\date{\today}% It is always \today, today,
             %  but any date may be explicitly specified

\begin{abstract}
Couple methods to demonstrate that barcodes can be used to copy a known connectome into a (small) neural circuit are explored.
%\begin{description}
%\item[Usage]
%Secondary publications and information retrieval purposes.
%\item[PACS numbers]
%May be entered using the \verb+\pacs{#1}+ command.
%\item[Structure]
%You may use the \texttt{description} environment to structure your abstract;
%use the optional argument of the \verb+\item+ command to give the category of each item. 
%\end{description}
\end{abstract}

\pacs{Valid PACS appear here}% PACS, the Physics and Astronomy
                             % Classification Scheme.
%\keywords{Suggested keywords}%Use showkeys class option if keyword
                              %display desired
\maketitle

%\tableofcontents

\section{\label{sec:level1}Introduction}

We want to play with the following problem. A connectome of certain size (Here I will be using 5 neurons) is well known. and we assume that barcodes can be produces. Our goal is to figure out a plausible physical situation where a network can form connections which is equivalent to the aforementioned connectome, using only barcodes.

\subsection{\label{sec:prob} Problem Description}
What we want is each cell to have a designated marker from a barcode. In order for this to happen, we can try to minimize the following function;

\begin{equation}
H = -\frac{1}{2}\sum_{\forall \mu \alpha}B_{\mu,\alpha}^{\gamma} = -\frac{1}{2} \sum_{\forall \mu \alpha} \left( \sum_{\nu \beta} x_{\mu\nu , \alpha \beta} + x_{\nu\mu , \beta\alpha} \right)^{\gamma}
\end{equation}

The notation used is as following;
\begin{itemize}
	\item H is the quantity to be maximized (Hamiltonian)
	\item $x_{\mu\nu , \alpha\beta}$ denotes the number of barcodes on the synapse from cell $\mu$ to cell $\nu$, with the barcode's pre-synaptic end marker being $\alpha$ and post-synaptic end marker being $\beta$. (Thus to find the number of markers pointing towards a cell, we add the barcodes on outgoing connections with the desired pre-synaptic end, and the ones on incoming connections with the desired post-synaptic end)
	\item $B_{\mu \alpha}$ denotes the number of markers of type $\alpha$ pointing to cell $\mu$
	\item $\gamma$ is some constant. Maximizing the function when $\gamma <= 1$ results in randomizing barcodes across neurons, while $\gamma > 1$ favors the situation where cells have a single marker pointing towards them in large numbers. (Increasing $\gamma$ increases this tendency for clumping)
\end{itemize}


\section{\label{sec:qp}Quadratic Programming}
 In the case where $\gamma=2$; the problem can be reformulated as a Quadratic Programming (QP) problem. QP is worked out, and has many algorithms to maximize a quadratic linear equation under constraints. When the desired case is expanded out, we have the following expression;
 
 \begin{equation}
 	H = - \left(\sum_{\mu \nu \alpha \beta \delta \sigma} \mathbf{x}_{\mu\nu,\alpha\beta}x_{\mu\sigma,\alpha\lambda} + \mathbf{x}_{\mu\nu,\alpha\beta}x_{\sigma\mu,\delta\alpha} \right)
 \end{equation}

Which can be reduced to the following linear equation;

\begin{equation}
	\begin{gathered}
		H = - \mathbf{x}_{\alpha\beta,\mu\nu}\mathbb{x}_{\delta\sigma,\rho\sigma} A^{\alpha\beta,\mu\nu}_{\delta\sigma,\rho\sigma} \\
		A^{\alpha\beta,\mu\nu}_{\delta\sigma,\rho\sigma} = \left[ J^{\beta\nu}_{\gamma\sigma}\delta_{\alpha\delta}\delta_{\mu\rho} + J^{\beta\nu}_{\delta\rho}\delta_{\alpha\gamma}\delta_{\mu\sigma} + J^{\alpha\mu}_{\gamma\sigma}\delta_{\beta\delta}\delta_{\nu\rho} + J^{\alpha\mu}_{\delta\rho}\delta_{\beta\gamma}\delta_{\nu\sigma} \right] \\
		J^{\alpha\beta}_{\gamma\omega} = 1 \\
		\delta_{\mu\nu} = 
		\begin{cases}
			1 & \mu = \nu \\
			0 & \mu \neq \nu
		\end{cases}
	\end{gathered}
\end{equation}

However, this is a rank 4 tensorial equation, for which QP algorithms are not developed. The most easily accessible QP algorithm minimizes H while satisfying the inequality;

\begin{equation}
	\begin{aligned}
		H = \frac{1}{2} \mathbf{x}^T Q \mathbf{x} + \mathbf{c}^T \mathbf{x} \\
		A \mathbf{x} \leq \mathbf{b}
	\end{aligned}
\end{equation}

which deals with vectors (rank 1 tensors). In order to deal with this problem, a new index will be introduced. While the details are given in Sectionn \ref{sec2:q}, the mapped index structure is as following; (the arithmetic operations correspond to the numeric values of the indices)

\begin{equation}
	\begin{gathered}
		x_{\mu\nu,\alpha\beta} = x_{I,J} = x_{\Lambda} \\
		I = (\mu ,\nu ) = n \times \mu + \nu \\
		\Lambda = (I, J) = n \times I + J 
	\end{gathered}
\end{equation}

\subsection{\label{sec2:q}Formulation of Q}
For re-indexing, the four indices are grouped into two; synaptic and barcode pairs. Since the number of neurons and markers are the same, index used to designate them are identical, and position of the indices designate if a given index is for synapse or barcode. The notation to be used is;

\begin{equation}
	\begin{aligned}
		\mathbf{x}_{\mu\nu,\alpha\beta} = \mathbf{x}_{IJ} \\
		M_{I\mu} = \begin{cases}
				1 & I = (\mu,_) \\
				0 & \textrm{otherwise}
			\end{cases} \\
			N_{I\mu} = \begin{cases}
			1 & I = (_,\nu) \\
			0 & \textrm{otherwise}
			\end{cases} \\
		B_{\mu\alpha} = \sum_{I, J} \mathbf{x}_{IJ} \left( M_{I\mu}M_{J\alpha} + N_{I\mu}N_{J\alpha} \right)
	\end{aligned}
\end{equation}

When the square is taken, and the sum over all neurons and barcodes are taken, H takes the form;

\begin{equation}
	\begin{split}
			H & = -\frac{1}{2} \sum_{\mu, \alpha} B_{\mu \alpha} \\
			  & = -\frac{1}{2} \sum_{I, J, P, Q} \mathbf{x}_{IJ}\mathbf{x}_{PS} \left[ \sum_{\mu, \alpha} K^{\mu\alpha}_{IJ,PS} \right]
	\end{split}
\end{equation}
\begin{equation}
	\begin{split}
		\sum_{\mu\alpha} K^{\mu\alpha}_{IJ,PS} = & \sum_{\mu\alpha} 
		\left[ M_{I\mu}M_{J\alpha} + N_{I\mu}N_{J\alpha} \right] \times \\
		& \left[ M_{P\mu}M_{S\alpha} + N_{P\mu}N_{S\alpha} \right] \\
		= & \sum_{\mu\alpha}  M_{I\mu}M_{J\alpha} M_{P\mu}M_{S\alpha} +\\
		  & M_{I\mu}M_{J\alpha} N_{P\mu}N_{S\alpha} \\
		  & N_{I\mu}N_{J\alpha} M_{P\mu}M_{S\alpha} \\
		  & N_{I\mu}N_{J\alpha} N_{P\mu}N_{S\alpha} \\
		= & (MM^T)_{IJ}(MM^T)_{PS} \\
		  & + (MM^T)_{IJ}(NN^T)_{PS} \\
		  & + (NN^T)_{IJ}(MM^T)_{PS} \\
		  & + (NN^T)_{IJ}(NN^T)_{PS} \\
		= Q_{IJPS}
	\end{split}
\end{equation}

Thus the transformation K is completely known. The forms of M and N are also deducible from the following Kronecker product;

\begin{equation}
	\begin{gathered}
		M = \mathbb{1} \otimes \vec{1} \\
		N = \vec{1} \otimes \mathbb{1} \\
		\vec{1} = 
			\begin{pmatrix}
				1 \\ 1 \\ \vdots \\ 1
			\end{pmatrix}_{n,1}
	\end{gathered}
\end{equation}

(when the IJ indexing is done by J=(J/n, J mod N), where N is the number of neurons) The problem is reduced to a rank 2 tensor equation, as the equation is second order in $\mathbf{x}_{IJ}$ which is rank 2. (From now on, Einstein notation will be adapted, as in repeated indices are summed over)

\begin{equation}
	H = -\frac{1}{2} \mathbf{x}_{IJ} W_{IJPS} \mathbf{x}_{PS}
\end{equation}

The index (I,J) can be joined together. Denoting the new index by $\Lambda = (I,J)$;
\begin{equation}
	\begin{split}
		H = & -\frac{1}{2} \mathbf{x}_{\Lambda} W_{\Lambda\Omega} \mathbf{x}_{\Omega} \\
			& -\frac{1}{2} \mathbf{x}^T W \mathbf{x} \\
			& \frac{1}{2} \mathbf{x}^T Q \mathbf{x}
	\end{split}
\end{equation}
With the introduction of new indexing, the problem is put in canonical form, at the cost of having $n^4$ dimensional space.

\begin{figure}[!ht]
	\centering
	\includegraphics[width=0.4\textwidth]{5neuronQ.eps}
	\caption{The Q matrix in the $5^4=625$ dimensional space}
\end{figure}

\subsection{\label{sec2:c}Formulation of $\vec{c}$}

It is desirable to have barcodes occupying synapses. A linear term can model this tendency, as it will enforce $\sum_{\Lambda} \mathbf{x}_\Lambda = \mathbf{c}^T \mathbf{x}$ to be the total number of barcodes in the system. Such a vector $\mathbf{c}$ would be;
\begin{equation}
	\mathbf{c}^T = C
	\begin{pmatrix}
		1 & 1 & \dots & 1
	\end{pmatrix}_{1,n^4}
\end{equation}
However, minimizing the quadratic term does force the $\mathbf{x}$ entries to grow, so unless the calculation is not helped by the $\mathbf{c}$ term, it is insignificant.

\subsection{\label{sec2:A}Formulation of Constraints; A and $\vec{b}$}

To begin with, a hard constraint is $\mathbf{x}_{\Lambda} \geq 0$. Besides this, we have another condition. The number of barcodes of one type is limited by our input, the connectivity mapping. Some barcodes might end up in the cytoplasm (or extracellular matrix), but there cannot be more barcodes in the synapses than the barcodes provided. The constraint is;

\begin{equation}
	\sum_{\mu\nu} \mathbf{x}_{\mu\nu\alpha\beta} + \mathbf{x}_{\mu\nu\beta\alpha} \leq \mathbf{b}_{\alpha\beta}
\end{equation}

The summation is symmetric, due to the barcodes having no information regarding their directionality in the synapse. Previous indexing scheme will introduce redundancy in dimensions, as without directionality, there are $n(n+1)/2$ distinct barcodes (as opposed to $n^2$ if the barcodes are directional). With the new types of indices running over from 1 to $n(n+1)/2$ (i, j, k will be used for these where)

\begin{equation}
	\begin{aligned}
		A_{i,\Lambda} \mathbf{x}_{\Lambda} \leq \mathbf{b}_i \\
		i = (\mu,\nu); \space \mu \leq \nu
	\end{aligned}
\end{equation}

\begin{figure}[!ht]
	\centering
	\includegraphics[width=0.4\textwidth]{5neuronA.eps}
	\caption{The constraint matrix A, for 5 neurons. Only columns 1 to $n^2$=25 are shown, out of $n^4$=625 columns. (Repeats for $n^2$=25 times)}
\end{figure}

The matrix is repetitive over the I index on $\Omega = (I,J)$; the I index is the synaptic index and each specific barcode must be summed over all possible synapses. 

The $\mathbf{b}$, in (i,j) notation, is derived from the original connection matrix. In $(\mu,\nu)$ notation, the $\mathbf{b}$ is; ($R_{\mu\nu}$ is the original connectome, marking synapses between pre-synaptic neuron $\mu$ and post-synaptic neuron $\nu$)

\begin{equation}
	\mathbf{b}_{\mu\nu} = C\left( R_{\mu\nu} + R_{\nu\mu} - R{\mu\nu}\delta_{\mu\nu} \right)
\end{equation}

The barcodes themselves do not have the directionality marker, therefore the $\mu \rightarrow \nu$
connection provides the same barcode as the $\nu \rightarrow \mu$ barcode.


%\subsection{\label{sec:level2}Second-level heading: Formatting}


%\subsubsection{Wide text (A level-3 head)}


%\subsection{\label{sec:citeref}Citations and References}

%\subsubsection{Citations}

%\paragraph{Syntax}
%The argument of \verb+\cite+ may be a single \emph{key}, 
%or may consist of a comma-separated list of keys.
%The citation \emph{key} may contain 
%letters, numbers, the dash (-) character, or the period (.) character. 
%New with natbib 8.3 is an extension to the syntax that allows for 
%a star (*) form and two optional arguments on the citation key itself.
%The syntax of the \verb+\cite+ command is thus (informally stated)
%\begin{quotation}\flushleft\leftskip1em
%\verb+\cite+ \verb+{+ \emph{key} \verb+}+, or\\
%\verb+\cite+ \verb+{+ \emph{optarg+key} \verb+}+, or\\
%\verb+\cite+ \verb+{+ \emph{optarg+key} \verb+,+ \emph{optarg+key}\ldots \verb+}+,
%\end{quotation}\noindent
%where \emph{optarg+key} signifies 
%\begin{quotation}\flushleft\leftskip1em
%\emph{key}, or\\
%\texttt{*}\emph{key}, or\\
%\texttt{[}\emph{pre}\texttt{]}\emph{key}, or\\
%\texttt{[}\emph{pre}\texttt{]}\texttt{[}\emph{post}\texttt{]}\emph{key}, or even\\
%\texttt{*}\texttt{[}\emph{pre}\texttt{]}\texttt{[}\emph{post}\texttt{]}\emph{key}.
%\end{quotation}\noindent
%where \emph{pre} and \emph{post} is whatever text you wish to place 
%at the beginning and end, respectively, of the bibliographic reference
%(see Ref.~[\onlinecite{witten2001}] and the two under Ref.~[\onlinecite{feyn54}]).
%(Keep in mind that no automatic space or punctuation is applied.)
%It is highly recommended that you put the entire \emph{pre} or \emph{post} portion 
%within its own set of braces, for example: 
%\verb+\cite+ \verb+{+ \texttt{[} \verb+{+\emph{text}\verb+}+\texttt{]}\emph{key}\verb+}+.
%The extra set of braces will keep \LaTeX\ out of trouble if your \emph{text} contains the comma (,) character.
%
%The star (*) modifier to the \emph{key} signifies that the reference is to be 
%merged with the previous reference into a single bibliographic entry, 
%%a common idiom in APS and\texttt{}

%\paragraph{The options of the cite command itself}
%Please note that optional arguments to the \emph{key} change the reference in the bibliography, 
%not the citation in the body of the document. 
%For the latter, use the optional arguments of the \verb+\cite+ command itself:
%\verb+\cite+ \texttt{*}\allowbreak
%\texttt{[}\emph{pre-cite}\texttt{]}\allowbreak
%\texttt{[}\emph{post-cite}\texttt{]}\allowbreak
%\verb+{+\emph{key-list}\verb+}+.

\end{document}
%
% ****** End of file apssamp.tex ******
